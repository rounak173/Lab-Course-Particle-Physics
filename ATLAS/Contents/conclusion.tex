%% LaTeX2e file `chap2'
%% generated by the `filecontents' environment
%% from source `main' on 2018/03/29.
%%
\chapter{Conclusion}\label{ch:conclusion}

A search was conducted for a heavy \(Z'\)-boson that decays into pairs of top quarks \(t \Bar{t}\) using data collected by the ATLAS detector at \(\sqrt{s} = 13\) TeV. Distributions were compared between the data and predictions from the Standard Model. At first, a large difference was observed between the data and predictions, indicated by a very low \(p\)-value \(1.7 \times 10^{-116}\) before considering uncertainties. However, after systematic uncertainties included, the agreement improved significantly, with the \(p\)-value increasing to 0.99998. This means no strong evidence for the presence of a \(Z'\)-boson was found.
\\

The high observable invariant mass of the \(t \Bar{t}\) pairs was used because it helps classify new particle signals from known background processes. \(Z'\) masses between about 410 GeV and 1800 GeV were excluded at a 95\% confidence level. For masses above 2 TeV, the current data is not sensitive enough to draw conclusions.
\\

Overall, limits on the existence of the \(Z'\)-boson have been placed through this analysis, and We, as master's students, have gained experience and understanding that will help us in future searches for new physics beyond the Standard Model.



