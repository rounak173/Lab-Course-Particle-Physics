\chapter{Dataset Description}\label{ch2}
Table \ref{tab:data_sim_samples} summarizes the datasets and simulated samples used in this analysis. It includes the recorded proton-proton collision data from 2016 as well as various Monte Carlo simulated backgrounds and signal samples.

\renewcommand{\arraystretch}{1.3} % increase row height
\begin{table}[h!]
\centering
\caption{pp collisions recorded in 2016 at \(\sqrt{s} = 13\) TeV}
\label{tab:data_sim_samples}
\begin{tabular}{@{}l p{8cm}@{}}
\toprule
\textbf{Sample} & \textbf{Description} \\ \midrule
Data & Events selected with a single-lepton trigger: electrons with \(E_T > 25\) GeV or muons with \(p_T > 25\) GeV. \\[6pt]

SM \(t\bar{t}\) background & Simulated at next-to-leading order (NLO) with parton shower. \\
                           % & Scaled to next-to-next-to-leading order (NNLO) + next-to-next-to-leading logarithmic (NNLL) cross sections. \\[6pt]

Other backgrounds & Single-top events using best available theoretical predictions. \\
                       & \(W\)+jets and \(Z\)+jets events \\
                       % , scaled using best available theoretical predictions. 
                       % & Diboson (\(WW\), \(WZ\), \(ZZ\)) events\\ , scaled using best available theoretical predictions. \\[6pt]

\(Z'\) signal & \\ 
Mass points: 500–3000 GeV & Narrow width assumption \((\Gamma / m \approx 1\%)\). \\
                       & SM-like coupling to \(t\bar{t}\) with full detector simulation. \\
\bottomrule
\end{tabular}
\end{table}








